\begin{abstract}
%\boldmath
% Named Data Network(NDN) is a new Internet architecture. Its change of the network layer also sheds light on the transport layer. The Data which comes back along the same path of the Interest natively acts as a barrier of the Explicit Congestion Notification (ECN) information to receiver.  Avoiding a congesting path can be achieved by adaptive forwarding which is a main feature of the NDN data plane. In this paper we implement an ECN transport mechanism in NDN, using the Data to carry ECN information. And we make use of network-wide information of the SDN controller  to design smart forwarding mechanism. Our simulations in ndnSim show that the ECN transport mechanism outperform TCP-style NDN transport mechanisms in link utilization, packet dropping and flow complete time. The network-view information of the SDN controller can optimize the adaptive forwarding in NDN. By joining the ECN transport mechanism and smart forwarding, the total flow complete time can be reduced.

Named Data Networking (NDN) is a new Internet architecture that shifts the communication paradigm from \emph{where} to \emph{what}. The data transport in NDN is completely driven by consumers via the Interest sending rate. An efficient transport control mechanism therefore should provide consumers with accurate network condition variation quickly. We in this paper presents an ECN-based (Explicit Congestion Notification) approach that explicitly piggybacks the network condition information in Interest and Data packets. As Data packet follows the exactly same path of Interest packet, the information carried back by Data packets accurately indicates the network condition. Consumers then proactively adjust Interest sending rates according to such accurate information to maximize the link utilization while avoiding congestion. We further propose a SDN-based smart forwarding mechanism that schedule traffic flows among available paths with global network information. Such a smart forwarding mechanism could further improve the resource utilization of the whole network. We finally evaluated the proposed approaches by packet-level simulation in ndnSIM. The results show that our approaches outperform TCP-style NDN transport mechanisms in link utilization, packet dropping and flow complete time.

\emph{Index Terms}-NDN, Transport mechanism, ECN, Adaptive forwarding
\end{abstract}
