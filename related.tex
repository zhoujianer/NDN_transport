%!TEX root = main.tex

\section{Related work}
Two kinds of congestion control mechanisms have been widely studied in the traditional network transport layer.

TCP-style. TCP-style mechanisms are widely used nowadays.  Every sender maintains a sending window. The sending window will additively increase if the RTT is below an estimated value, otherwise window will multiplicatively decrease\cite{TCP}. TCP-style mechanism is very easy to implement, just on the host. But the utilization is low in high bandwidth-delay network because the slow start principle and it is unfair when long flow competes with short flow.

ECN-style. ECN-style mechanism uses ECN information to sense the condition of the network, such as XCP\cite{XCP}, RCP\cite{RCP}. The senders react its sending rate according the ECN information. Compared with TCP-style, who use implicit congestion information, ECN-style can effectively use the network resource and achieve fairness. Although our Interest sending rate's design principle is similar with RCP, our design process is different. In this paper we use \emph{R(t)} to control the receiver's sending rate, but RCP use the \emph{R(t)} to control the sender's packet sending rate. We also have to consider the size of Data, and in RCP the packet's size has not relationship with the design process.

Some transport mechanisms have been proposed for NDN. Most of them is TCP-style. Some researchers use NDN's features, such as the NACK, to improve the TCP-style mechanism.

Receiver driven TCP-style. Different with TCP/IP's push principle, NDN is a pull-way network architecture. So it is the receivers no longer the senders who dominant the traffic of network. Most of NDN transport mechanisms now proposed is receiver-driven. Receivers adjust its Interest sending window according to the RTT of the coming back Data, using the AIMD and slow start principle. Contug\cite{Contug} is the first TCP-style receiver-driven mechanism in NDN, following the basic principle in TCP. It just changes the congestion-controller from sender to receiver.  In\cite{Flow}, Sara, etc. use the prefix of the name to separate the traffic into different flows. Each flow uses TCP-like mechanism to control the congestion. Routers fairly share the bandwidth to each flow, and using optimal buffer algorithm to improve the performance. In NDN, in-network cache is available. Content can be get from different providers or routers. Different locations of content may result in multipath in transport layer. In \cite{Multipath}, Givonaan, etc. deal with the multipath problem in NDN, using similar way as the MTCP.

Interest NACK. In NDN the Interest is much shorter than Data. Intuitively, it is much more ��resource-saving�� if we drop Interest instead of Data when congestion happens. In HR-ICP\cite{shape}, routers shape the Interest hop-by-hop to deal with or avoid congestion. Each router shapes Interest just by itself, according the bandwidth it can supply to coming Data. In \cite{improveshape} Wang, etc. improve the Interest shaping mechanism. Once an Interest is shaped, router sends NACK Interest back to the receiver to inform that congestion has happened. The NACK which sends back to the receiver is useful information to inform the receiver that congestion happens. But it is still implicit information, does not inform the degree of congestion. The receiver simply cut half of the Interest sending window when it receives the congestion signal. That will reduce the link utilization as in TCP. And the shaping mechanism, no matter Interest or Data will cause retransmit. Retransmit will increase the flow complete time.

To fit the pull principle of NDN, all of these above mechanisms use the receiver to deal with congestion. But they still follow the TCP's slow start and AIMD implicit congestion principle. They also have the low utilization and unfairness problems, similar with TCP\cite{NDNanalysis}.
