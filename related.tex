%!TEX root = main.tex

\section{Related work}

\label{sec:related}

% Two kinds of congestion control mechanisms have been widely studied in the traditional network transport layer.

% TCP-style. TCP-style mechanisms are widely used nowadays.  Every sender maintains a sending window. The sending window will additively increase if the RTT is below an estimated value, otherwise window will multiplicatively decrease\cite{TCP}. TCP-style mechanism is very easy to implement, just on the host. But the utilization is low in high bandwidth-delay network because the slow start principle and it is unfair when long flow competes with short flow.

% ECN-style. ECN-style mechanism uses ECN information to sense the condition of the network, such as XCP\cite{XCP}, RCP\cite{RCP}. The senders react its sending rate according the ECN information. Compared with TCP-style, who use implicit congestion information, ECN-style can effectively use the network resource and achieve fairness. Although our Interest sending rate's design principle is similar with RCP, our design process is different. In this paper we use \emph{R(t)} to control the receiver's sending rate, but RCP use the \emph{R(t)} to control the sender's packet sending rate. We also have to consider the size of Data, and in RCP the packet's size has not relationship with the design process.

Congestion control mechanisms in TCP/IP network can be grouped into two categories:

\emph{Congestion Window} Congestion window mechanism is widely deployed nowadays. Each end in the communication maintains a congestion window, which will additively increase if the RTT is below the estimation value, otherwise multiplicatively decrease, known as AIMD\cite{TCP}. Congestion window mechanism is easy to deploy since it only requires modification in end hosts. However, congestion window results to low utilization in high bandwidth-delay networks because of the slow start function. Moreover, it results in unfairness for short flows when mixed with long flows.

\emph{Explicit Congestion Notification (ECN)} ECN mechanism notifies network condition by explicit information, such as in XCP\cite{XCP}, RCP\cite{RCP}. End host adjusts its transmission rate according to the ECN information it has received. Compared with implicit congestion information in TCP, ECN can effectively utilize network resource and achieve fairness. However, RCP use congestion information to control server's transmission rate, which is contrary to the consumer-driven nature in NDN. Besides, RCP does not consider the impact of Data size on the estimation of flow size, which is also essential in congestion control in NDN.

% Those works above failed to achieve considerable performance in congestion control in NDN because they do not utilize its specific property, such as consumer-driven, one-to-one mapping between Interest and Data.

Some transport mechanisms have been proposed for NDN. Most of them is TCP-style. Some researchers use NDN's features, such as the NACK, to improve the TCP-style mechanism.

\emph{Receiver driven TCP-style.} Different with TCP/IP's push principle, NDN is a receiver-driven network architecture. In NDN, receivers control transmission rate by adapting the rate of sending Interest. Thus, most of proposed transport mechanisms in NDN are receiver-driven. Receiver adjust its Interest sending rate according to the RTT of the coming back Data, using the AIMD and slow start principle. Contug et al.\cite{Contug} propose transport mechanism in NDN which mixes receiver-driven mechanism and the basic principles in TCP. It just changes the congestion-controller from sender to receiver. In\cite{Flow}, Sara et al. separate the traffic into different flows according to name prefix. Each flow uses TCP-like mechanism to control the congestion. Routers fairly allocate the bandwidth to each flow, and use optimal buffer algorithm to improve the performance. In NDN, due to in-network cache, content can be retrieved from different providers or routers. Different locations of content may result in multipath in transport layer. In \cite{Multipath}, Givonaan et al. deal with the multipath problem in NDN, using similar way as the MTCP.

\emph{Interest NACK.} In NDN an Interest is much smaller than a Data. Intuitively, it is much more ``resource-saving'' if we drop Interest instead of Data when congestion happens. In HR-ICP\cite{shape}, routers shape the Interest hop-by-hop to handle or prevent congestion. Each router shapes Interest just by itself, according the bandwidth it can supply to incoming Data. In \cite{improveshape} Wang et al. improve the Interest shaping mechanism as follows. Once an Interest is shaped, router sends NACK back to the receiver to inform that congestion has occurred. Although NACK implies congestion, it does not inform how severe the congestion is. The receiver simply cuts half of the Interest sending window when it receives NACK, which will reduce the link utilization significantly as in TCP. Furthermore, the shaping mechanism, no matter Interest or Data will cause retransmission, which will increase the flow complete time.

To fit the pull principle of NDN, all of these above mechanisms use the receiver to deal with congestion. But they still follow the TCP's slow start and AIMD implicit congestion principle. They also have the low utilization and unfairness problems, similar with TCP\cite{NDNanalysis}.
