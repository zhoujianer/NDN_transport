%!TEX root = main.tex

\section{Related work}
% no \IEEEPARstart
\label{sec:related}

Congestion control is a hot topic in both TCP/IP network and NDN network. We first survey the mechanism in TCP/IP network:

\textbf{Congestion Window.} Congestion window mechanism is widely deployed nowadays. Each end in the communication maintains a congestion window, which additively increases if the RTT is below the estimated value, otherwise multiplicatively decrease. This mechanism is known as AIMD\cite{TCP}. Congestion window mechanism is easy to deploy since it only requires modification in end hosts. However, congestion window results in low utilization in high bandwidth-delay networks because of the slow start function. Moreover, it results in unfairness for short flows when mixed with long flows\cite{tcpdeadline}.

\textbf{Explicit Congestion Notification (ECN)}. ECN mechanism notifies network condition by explicit information, such as in XCP\cite{XCP}, RCP\cite{RCP}. Router generates ECN information according to network condition and send it to end host. End host adjusts its transmission rate according to the ECN information it has received. Compared with implicit congestion information in TCP, ECN can effectively utilize network resource and achieve fairness. However, RCP uses congestion information to control server's transmission rate, which is contrary to the consumer-driven nature in NDN. Besides, RCP does not consider the impact of Data size on the estimation of flow size, which is also essential in congestion control in NDN.

Several transport mechanisms have been proposed for NDN. Most of them is TCP-style. Some researchers use NDN's features, such as the NACK, to improve the TCP-style mechanism.

\textbf{Receiver driven TCP-style}. Most of proposed transport mechanisms in NDN are receiver-driven. Receiver adjusts its Interest sending rate according to the RTT of the coming back Data, using the AIMD and slow start principle\cite{NDNanalysis}. Contug et al.\cite{Contug} propose transport mechanism which mixes receiver-driven mechanism and the basic principles in TCP. It changes the congestion-controller from sender to receiver. In\cite{Flow}, Sara et al. separate the traffic into different flows according to name prefix. Each flow uses TCP-like mechanism to control the congestion. Routers fairly allocate the bandwidth among flows, and use optimal buffer algorithm to improve the performance. In NDN, due to in-network cache, content can be retrieved from different providers or routers. Different locations of content may result in multipath in transport layer. In \cite{Multipath}, Givonaan et al. deal with the multipath problem in NDN, using similar way as the MTCP.

\textbf{Interest NACK}. An Interest is much smaller than a Data. Intuitively, it is much more ``resource-saving'' if we drop Interest instead of Data when congestion happens. In HR-ICP\cite{shape}, routers shape the Interest hop-by-hop to handle or prevent congestion. Each router shapes Interest by itself, according to the bandwidth it can supply to incoming Data. In \cite{improveshape} Wang et al. improve the Interest shaping mechanism. Once an Interest is shaped, router sends NACK back to the receiver to inform that congestion has occurred. Although NACK implies congestion, it does not inform how severe the congestion is. The receiver simply cuts half of the Interest sending window and begin slow-start when it receives NACK. Although cutting the window to half and slow-start can deal with congestion, they reduce the link utilization. Furthermore, the shaping mechanism, no matter Interest or Data, causes retransmission, and that increases the flow complete time. 

\textbf{Adaptive forwarding}. Adaptive forwarding is a main feature of NDN data plain. In TCP/IP, forwarding table completely follows the route table without any adaptability. However in NDN, during the forwarding process, router can adaptively choose a forwarding interface from several available paths according the network situation. In \cite{Adaptive} Cheng, et al. make use of the adaptive forwarding mechanism to design a hop-by-hop congestion control mechanism. Routers adaptively forward the Interest to another interface when it detects that the next hop has been congested. However such just one hop detection is not enough if the congestion happens on later link, because the one hop detection cannot sense the congestion on later link immediately. 