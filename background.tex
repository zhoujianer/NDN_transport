%!TEX root = main.tex

\section{Background}

\label{sec:bg}

% We recall some features of NDN that we will use to design our transport mechanism. The full description of NDN can be found in NDN paper\cite{NDN}\cite{Adaptive}.

NDN adopts transport mechanism \cite{NDN, Adaptive} which is distinct from TCP/IP network or other ICN proposals. In NDN, consumer issues one Interest to retrieve only one piece of Data. Thus consumer could control the transmission rate by adapting the number of issuing Interests. On receiving an Interest, routers maintain an entry for it in its PIT to ensure Data is forwarded back exactly along the path that Interest is transmitted. Such a ``reverse path'' feature enables that Data could carry Explicit Congestion Notification (ECN) information back to consumers.

Intermediate routers could also facilitate shaping flows by adaptive forwarding \cite{Adaptive}. When forwarding an Interest, NDN router adaptively selects an interface among the multiple available ones according to the network condition. If there is no satisfying interface for forwarding, NDN router responds to downstream router with explicit NACK. Thus, downstream routers react to network congestion in a hop-by-hop way. This is much easier than that in TCP/IP because router-assistant congestion mechanism needs the help of route protocol\cite{selfish}. However only one hop detection is not enough if the congestion happens on links several hop away from consumers, because the one hop detection cannot sense it immediately. Since SDN\cite{SDN} could be used to gather network-wide congestion information, a SDN-based solution is promising to overcome the limit of the one-hop solution.

% TODO what can we do based on such a flow control?

% Data feedback. In NDN, consumers send Interest to request Data. The router uses Content Store, Pending Interest Table(PIT) and Forwarding Information Base(FIB) to return , record and forward the Interest respectively. For consumers, one Interest pulls back exactly one Data. As the PIT maintains every Interest's state, such as the coming and leaving entry, the Data will come back along exactly the same path with Interest. Such ``same path" feedback let the Data natively act as carrier to bring back the ECN information to consumers.

% Adaptive forwarding. Adaptive forwarding is a main feature of NDN. In TCP/IP, forwarding table completely follows the route table without any adaptability, and there is just one path to the destination in the route table. However in NDN, during the forwarding process, router can adaptively choose a forwarding interface from several available paths according the network situation. In \cite{Adaptive} Cheng, etc. make use of the adaptive forwarding mechanism to design a hop-by-hop congestion control mechanism. Routers adaptively forward the Interest to another interface (if available) when it detects that the next hop has been congested. Such adaptive forwarding way helps the network solve congestion much easier than the route assistant congestion in TCP/IP, because the route-assistant congestion mechanism needs the help of route protocol\cite{selfish}. However such just one hop detection is not enough if the congestion happens on later link, because the one hop detection cannot sense it.  SDN-style solution can get the network-wide information\cite{SDN}. And such network-wide information is a promise way to overcome the limit of the just one hop information.
