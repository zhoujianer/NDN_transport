%!TEX root = main.tex

\section{Introduction}
% no \IEEEPARstart

Internet consumers typically care more about \emph{what} information they want, instead of \emph{where} it is located. However, the Internet architecture is initially designed for point-to-point communication. This mismatch between user demand and network infrastructure makes Internet applications struggling with the gap between where and what. Several clean slate network architectures have been proposed to address this problem. Among those proposals, NDN\cite{NDN} assigns each data a unique name for addressing and caching, which is compatible with Internet demand and exhilarate information dissemination greatly.


As NDN adopts simple best-effort packet transport between neighboring NDN nodes, congestion might occur and packets could be dropped due to congestion. Therefore transport control is essential for effective and efficient data transmission. Some TCP-style transport control mechanisms in NDN have been proposed, such as ICP\cite{ICP}, CCTCP\cite{CCTCP} and HR-ICP\cite{shape}. These TCP-style proposals still suffer from the same problems as that in TCP, e.g. low link utilization, high packet dropping rate. Explicit Congestion Notification (ECN) carries the network resource usage condition to consumers, and consumers can adjust how to use the network resource ultimately\cite{XCP}. So ECN is a promising technique to achieve high link utilization. NDN adopts transport mechanism \cite{NDN, Adaptive} which is distinct from TCP/IP network or other ICN proposals. In NDN, consumer issues one Interest to retrieve only one piece of Data. On receiving an Interest, routers maintain an entry for it in its PIT to ensure Data is forwarded back exactly along the path that Interest is transmitted. Such a ``reverse path'' feature enables that Data could carry ECN information back to consumers.

Intermediate routers could also facilitate shaping flows by adaptive forwarding \cite{Adaptive}. When forwarding an Interest, NDN router adaptively selects an interface among the multiple available ones according to the network condition. If there is no satisfying interface for forwarding, NDN router responds to downstream router with explicit NACK \cite{Adaptive}. Thus, downstream routers react to network congestion in a hop-by-hop way. This is much easier than that in TCP/IP because router-assistant congestion mechanism needs the help of route protocol\cite{selfish}. However only one hop detection is not enough if the congestion happens on links several hops away from consumers, because the one hop detection cannot sense it immediately. Since SDN\cite{SDN} could be used to gather network-wide congestion information, a SDN-based solution is promising to overcome the limit of the one-hop solution.

As far as we know, there's still no ECN-style congestion mechanism proposed for NDN and little research about congestion avoidance mechanism by utilizing adaptive forwarding has been investigated. In this paper, by utilizing the feature in NDN that there's one-to-one mapping of Interest-Data and Data is forwarded back along exactly the path Interest is transmitted, in Data packet we introduce an ECN field, which contains ECN information to notify consumer or down-stream routers to adapt their transmission rate. Furthermore, we also exploit the power of network-wide information by SDN\cite{SDN} to design a smart forwarding mechanism. By combining ECN and smart forwarding mechanism, we could not only improve bandwidth utilization on single links but also the resource utilization in the whole network. In summary, our main contributions are:

\begin{enumerate}
	\item[1.] We propose an ECN-based transport mechanism in NDN.
	\item[2.] We utilize SDN technique to design smart forwarding mechanism to fully exploit the resource in the whole network.
	\item[3.] Packet-level simulation shows that by joining ECN and smart forwarding in NDN, the bottleneck link utilization improved from 10\%-20\%, and there is almost no dropping packets.
\end{enumerate}


The rest of the paper is organized as followed: Sec. \ref{sec:related} discusses related works. In Sec. \ref{sec:rationale} we discuss the design rationale. Sec. \ref{sec:design} introduces the ECN congestion mechanism and the smart forwarding. In Sec. \ref{sec:simulation}, we demonstrate the effectiveness of our mechanism. Finally Sec. \ref{sec:conclude} concludes the paper.
