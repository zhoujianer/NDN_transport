%!TEX root = main.tex

\section{Introduction}
% no \IEEEPARstart
People now using network care more about ``what" they can get from network. However TCP/IP, the network architecture of Internet, is initially designed based on the principle of ``where" to connect the users. This mismatch between the user-demand and network-principle makes the network difficult to satisfy the users. To overcome this, some clean-state network architecture have been proposed. NDN is one of such clean-state architecture\cite{NDN}.In NDN, users just send Interest into the network, and the network will return the corresponding Data to it, unnecessary to care about where to get the Data.

As the NDN network layer is still based on the best-effort transmit model, it is necessary to use transport control to guarantee effective transfer. Some TCP-style transport control mechanisms have been proposed for NDN, such ICP\cite{ICP}, CCTCP\cite{CCTCP} and HR-ICP\cite{shape}. However these TCP-style transport mechanisms in NDN still have same problems as TCP/IP, such as low link utilization, and high packets dropping rate. Explicit congestion notification(ECN) is a promise way to achieve high link utilization\cite{XCP} . Adaptive forwarding is a new feature of the NDN\cite{Adaptive}. Adaptive forwarding let router choose a suitable path according the network situation. If the router sense the link has been congested, then it can choose another path. It is completely different with TCP/IP, as in TCP/IP the forwarding process is strictly followed the route table. By make use of the adaptive forwarding, router can deal with the network congestion quickly.

But by now, as what we have known, there is not ECN-style congestion mechanism proposed in NDN and there is little research about the congestion avoiding mechanism making use of the adaptive forwarding. In this paper we use the Data(which come back along the same way of Interest) to carry the ECN information, and design an ECN transport mechanism for NDN. We also make use of the SDN's network-wide information to choose a suitable forwarding interface\cite{SDN}. Joining the ECN and smart-adaptive forwarding mechanism, we can not only improve single link bandwidth utilization but also the whole network resource utilization. Summary, our main contributions are:
\begin{enumerate}
\item[1.]First achieve an ECN transport mechanism in NDN.

\item[2.]First use SDN-style control information to design smart forwarding mechanism to ultimately use the whole network resource.

\item[3.] Packet-level simulation shows that joining ECN and smart forwarding in NDN can improve link utilization and the total flow complete time.
\end{enumerate}
The rest of the paper is organized as followed: in Sec. \uppercase\expandafter{\romannumeral 2} we will introduce the NDN's feature that we will use to design our transport mechanism. In Sec. \uppercase\expandafter{\romannumeral 3} we will discuss the design rationale. In Sec. \uppercase\expandafter{\romannumeral 4} we will introduce the ECN congestion mechanism and the SDN-style smart forwarding. In Sec. \uppercase\expandafter{\romannumeral 5}, we demonstrate the effectiveness of our mechanism. In Sec. \uppercase\expandafter{\romannumeral 6}, we will introduce related work. Finally Sec. \uppercase\expandafter{\romannumeral 7} concludes the paper.


