%!TEX root = main.tex

\section{Introduction}
% no \IEEEPARstart

Internet consumers typically care more about \emph{what} information they want, instead of \emph{where} it is located. However, the Internet architecture is initially designed for point-to-point communication. This mismatch between user demand and network infrastructure makes Internet applications struggling with the gap between where and what. NDN (Named Data Networking) \cite{NDN}, a clean slate network architecture, is proposed to address this problem. It shifts the communication paradigm to data chunks, where each data chunk is uniquely identified by a name for addressing and caching.

The data transmission in NDN is consumer-driven. Consumer sends \emph{Interest} to express the intent on content. The transmission is hop-by-hop and intermediate routers keep states of the transmission. Data packet follows the reverse path of Interest. Intermediate routers could use adaptive forwarding \cite{Adaptive} to select the next-hop among the multiple available ones according to the network condition and policies. If there is no satisfying interface for forwarding deal to congestion, the router responds to downstream routers with explicit NACK \cite{Adaptive}. This could be viewed by the downstream routers as an indication of congestion. The consumer slows down the Interest sending rate when receive NACK from the upstream router. As one Interest packet retrieves one Data packet, slowing down the Interest sending rate would essentially decrease the data traffic. While simple, such a reactive congestion control mechanism cannot react to the congestion quickly on links several hops away from consumers. Thus, proactive transport control that leverages the information of whole path is highly demanded for NDN, in order to achieve effective and efficient data transmission.

Existing transport control mechanisms in NDN, such as ICP\cite{ICP}, CCTCP\cite{CCTCP} and HR-ICP\cite{shape}, adopts the basic idea of TCP in NDN. They still use reactive congestion control and therefore suffer from low link utilization and high packet dropping rate. In fact, the fact that Data packet follows the exact path of Interest provides us an opportunity for consumer to perceive the congestion in the path and consequently adjust the Interest sending rate. We in this paper explore this design space by using Explicit Congestion Notification (ECN) \cite{XCP}. ECN provides consumers with the network resource usage condition by explicitly carrying such information in the response packet. In such a way, consumers can proactively adjust the Interest sending rate according to network condition. We implement ECN-based transport control by only piggybacking very limited information which could be updated by intermediate routers.

The transmission efficiency of the network could be further improved by global coordination of traffic flows. To this end, we further propose a SDN-based mechanism, which employs controllers to collect the network condition of the whole network. The availability of global information enables the scheduling of flows among different paths, which in turn could maximize the network resource utilization and the transmission performance as well. In detail, we make three technical contributions as follows.

\begin{enumerate}
	\item[1.] We propose an ECN transport mechanism in NDN, which can ultimately use link bandwidth using ECN information to control consumers' Interest sending rate.
	\item[2.] Based on the ECN transport mechanism, we further utilize SDN to design smart forwarding mechanism to fully exploit the whole network resource.
	\item[3.] Packet-level simulation shows that by joining ECN and smart forwarding, the bottleneck link utilization improves 10\%-20\%, and there is almost no dropping packets.
\end{enumerate}


The rest of the paper is organized as followed: Sec. \ref{sec:related} discusses related works. Sec. \ref{sec:design} introduces the ECN-base transport mechanism and the smart forwarding. In Sec. \ref{sec:simulation}, we demonstrate the effectiveness of our mechanism. Finally Sec. \ref{sec:conclude} concludes the paper.
